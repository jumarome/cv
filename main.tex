%% Inicio del archivo `template-es.tex'.
%% Copyright 2006-2013 Xavier Danaux (xdanaux@gmail.com).
%
% This work may be distributed and/or modified under the
% conditions of the LaTeX Project Public License version 1.3c,
% available at http://www.latex-project.org/lppl/.


\documentclass[11pt,a4paper,sans]{moderncv}   % opciones posibles incluyen tamaño de fuente ('10pt', '11pt' and '12pt'), tamaño de papel ('a4paper', 'letterpaper', 'a5paper', 'legalpaper', 'executivepaper' y 'landscape') y familia de fuentes ('sans' y 'roman')

% temas de moderncv
\moderncvstyle{casual}                        % las opciones de estilo son 'casual' (por omision),'classic', 'oldstyle' y 'banking'
\moderncvcolor{blue}                          % opciones de color 'blue' (por omision), 'orange', 'green', 'red', 'purple', 'grey' y 'black'
%\renewcommand{\familydefault}{\sfdefault}    % para seleccionar la fuente por omision, use '\sfdefault' para la fuente sans serif, '\rmdefault' para la fuente roman, o cualquier nombre de fuente
\nopagenumbers{}                             % elimine el comentario para suprimir la numeracion automatica de las paginas para CVs mayores a una pagina

% codificacion de caracteres
\usepackage[utf8]{inputenc}                  % reemplace con su codificacion
%\usepackage{CJKutf8}                         % si necesita usa CJK para redactar su CV en chino, japones o coreano

% ajustes para los margenes de pagina
\usepackage[scale=0.75]{geometry}
%\setlength{\hintscolumnwidth}{3cm}           % si desea cambiar el ando de la columna para las fechas



% datos personales
\name{Juan Manuel}{Romero}
%\title{T\'itulo del CV (opcional)}                   % dato opcional, elimine la linea si no desea el dato
\social[linkedin][ec.linkedin.com/in/jmanuelromero]{jmanuelromero}
\social[github][github.com/jumarome] {jumarome}
\address{Urbanización Aquamarina solar 131}{Samborondón} % dato opcional, elimine la linea si no desea el dato
\phone[mobile]{+~(593)~9~94695642}                     % dato opcional, elimine la linea si no desea el dato
\phone[fixed]{+~(593)~4~2820316}                      % dato opcional, elimine la linea si no desea el dato                      % dato opcional, elimine la linea si no desea el dato
\email{jmanuelromero93@gmail.com}                                 % dato opcional, elimine la linea si no desea el dato
%\homepage{www.juanromero.ec}                         % dato opcional, elimine la linea si no desea el dato
%\extrainfo{}                    % dato opcional, elimine la linea si no desea el dato
\photo[64pt][0.4pt]{img/photo}                         % '64pt' es la altura a la que la imagen debe ser ajustada, 0.4pt es grosor del marco que lo contiene (eliga 0pt para eliminar el marco) y 'picture' es el nombre del archivo; dato opcional, elimine la linea si no desea el dato
                    % dato opcional, elimine la linea si no desea el dato


% para mostrar etiquetas numericas en la bibliografia (por omision no se muestran etiquetas), solo es util si desea incluir citas en en CV
%\makeatletter
%\renewcommand*{\bibliographyitemlabel}{\@biblabel{\arabic{enumiv}}}
%\makeatother

% bibliografia con varias fuentes
%\usepackage{multibib}
%\newcites{book,misc}{{Libros},{Otros}}
%----------------------------------------------------------------------------------
%            contenido
%----------------------------------------------------------------------------------
\begin{document}
%\begin{CJK*}{UTF8}{gbsn}                     % para redactar el CV en chino usando CJK
\renewcommand*\namefont{\fontsize{22}{48}\selectfont}
\maketitle

\section{Perfil}
Soy Ingeniero en Ciencias Computacionales, graduado en la \textbf{ESPOL} y actualmente me encuentro cursando una Maestría en Administración de Empresas en ESPAE. Me considero una persona responsable,puntual,comunicativa y con liderazgo. La tecnología y la programación son mis dos grandes pasiones, por lo que siempre estoy actualizando mis conocimientos,leyendo sobre nuevas tecnologías y poniéndolas en practica en proyectos personales. Poseo una amplia experiencia en el  desarrollo de aplicaciones web utilizando  frameworks de diferentes lenguajes de programación como \textbf{Java,PHP,Python}.
He desarrollado aplicaciones empresariales utilizando tecnología \textbf{JEE7} ( Java Enterprise Edition v7) y Spring Framework(v4).
He sido responsable de instalar y administrar servidores de Aplicaciones JEE como Wildfly y Glassfish en sistemas linux en Amazon EC2.
Poseo una amplia experiencia en el motor de reportes Jasper Reports, que va desde integrarlo en aplicaciones web , hasta integrarlo en \textbf{Odoo}(Plataforma de Aplicaciones de Negocios)





\section{Datos Personales}
\cvitem{Nombres}{Juan Manuel}
\cvitem{Apellidos}{Romero Santamaría}
\cvitem{Nacionalidad}{Ecuatoriana}
\cvitem{Identificación}{0917914160}
\cvitem{Estado Civil}{Soltero}
\cvitem{Fecha de Nacimiento}{6 de Octubre de 1989}


\section{Idiomas}
\cvitemwithcomment{Español}{Nativo}{}
\cvitemwithcomment{Ingles}{Intermedio}{}


\section{Formaci\'on acad\'emica}
\cventry{1996--2001}{Primaria}{Centro de Estudios Espíritu Santo}{Guayaquil}{}{}  % Los argumentos del 3 al 6 pueden permanecer vacios
\cventry{2002--2007}{Secundaria}{Colegio Particular Experimental Politécnico}{Guayaquil}{}{}
\cventry{2008--2014}{Universitaria}{Escuela Superior Politécnica del Litoral}{Guayaquil}{Ingeniero en Ciencias Computacionales}{}


\clearpage
\section{Conocimientos Profesionales}
\subsection{Lenguajes de Programacion}

Java es el lenguaje de programación que mas domino, he trabajado con el por mas de 8 años, tiempo en el cual he aprendido los diferentes API's que este lenguaje ofrece, tanto en su version standard como en su version Enterprise (Java SE y Java EE). 
Dentro de las especificaciones Java que domino se encuentran las siguientes:
\newline{}%

\cvlistitem{Java Servlet 3.1 (JSR 340)}
\cvlistitem{JavaServer Pages 2.3 (JSR 245)}
\cvlistitem{Java Server Faces 2.2 (JSR 344)}
\cvlistitem{Expression Language 3.0 (JSR 341)}
\cvlistitem{JSTL 1.2 (JSR 52)}
\cvlistitem{Java API for JSON Processing (JSR 353)}
\cvlistitem{Contexts and Dependency Injection for Java 1.1 (JSR 346)}
\cvlistitem{Bean Validation 1.1 (JSR 349)}
\cvlistitem{Enterprise JavaBeans 3.2 (JSR 345)}
\cvlistitem{Interceptors 1.2 (JSR 318)}
\cvlistitem{Java Persistence 2.1 (JSR 338)}
\cvlistitem{Java Transaction API (JTA) 1.2 (JSR 907)}
\cvlistitem{JavaMail 1.5 (JSR 919)}
\cvlistitem{Java API for RESTful Web Services (JAX-RS) 2.0 (JSR 339)}
\cvlistitem{Java API for XML-Based Web Services (JAX-WS) 2.2 (JSR 224)}
\cvlistitem{Java Architecture for XML Binding (JAXB) 2.2  (JSR 222)}
\cvlistitem{Java Database Connectivity 4.0  (JSR 221)}

Aparte de estas especificaciones, he trabajado con librerias de terceros como Primefaces,Spring Framework,Apache Shiro y Hibernate.También he desarrollado aplicaciones Android y las he comunicado mediante servicios web REST con mis proyectos de Java, Ademas de esto siempre uso maven para construir  y administrar las dependencias de mis proyectos Java
\newline{}%.

\cvitem{\textbf{Javascript}}{Especificación ECMAScript 5 ,Node.js,npm,grunt y AngularJS v1.X }
\cvitem{\textbf{Python}}{Aplicaciones de consola y Aplicaciones web con Flask y Odoo}
\cvitem{\textbf{PHP}}{Symfony 2.0 ,Bolt CMS,Creacion de sitios web con Drupal V8 }
\cvitem{\textbf{C}}{Aplicaciones en General \newline}

\subsection{Sistemas de Gestión de Bases de Datos}
\cvitem{\textbf{PostgreSQL}}{Instalacion en Linux ,Métodos de Autenticacion, Respaldo y Restauración}
\cvitem{\textbf{MySQL}}{Instalación en Linux, Seguridad, Respaldo y Restauración\newline}

\clearpage
\subsection{Ingenieria de Software}
\cvlistitem{Fundamentos de la planificación de proyectos de Software}
\cvlistitem{Técnicas de captura de requerimientos funcionales}
\cvlistitem{Estimación de Esfuerzo}
\cvlistitem{Grupos de Trabajo y asignación de recursos}
\cvlistitem{Monitoreo de Proyectos de Software}
\cvlistitem{Administracion de Riesgos}
\cvlistitem{Modelamineto de Software en UML}
\cvlistitem{Patrones de Diseño}
\cvlistitem{Estandares y documentación de Pruebas de Software\newline}



\subsection{Servidores Web y Servidores de Aplicaciones}
\cvlistitem{Apache Tomcat 7}
\cvlistitem{Apache}
\cvlistitem{GlassFish 4}
\cvlistitem{Wildfly 10}
\cvlistitem{Nginx\newline}


\subsection{Entornos de Desarrollo}
\cvlistitem{Netbeans}
\cvlistitem{PHPStorm}
\cvlistitem{IntelliJIDEA}
\cvlistitem{Webstorm}
\cvlistitem{Eclipse}
\cvlistitem{PyCharm\newline}

\subsection{Sistemas Operativos}
\cvlistitem{Windows 7}
\cvlistitem{OSX}
\cvlistitem{Ubuntu Server, Ubuntu Desktop  12.04 14.04 }

\subsection{Miscelaneos}
\cvlistitem{Diseño de reportes en JasperReports e integracion con JSF}
\cvlistitem{Framework de Seguridad Java Apache Shiro y Spring Security}
\cvlistitem{pgAdmin, phpMyAdmin}
\cvlistitem{Diseño Web utilizando Bootstrap 3,Foundation 5 y Sass}
\cvlistitem{Automatizacion de tareas con Grunt y Gulp }
\cvlistitem{Motor de plantillas FreeMarker para Java}
\cvlistitem{Creacion de Sitios Web utilizando Bolt CMS y Drupal 8}
\cvlistitem{Desarollo de Modulos para Odoo(OpenERP) V7 y V8}
\cvlistitem{Implementacion de Sistema de Gestion Documental Alfresco 5}
\cvlistitem{Elaboración de Documentos con \LaTeX}
\cvlistitem{Administracion de Servidores Linux en Amazon EC2\newline}





\section{Experiencia Profesional}
\cventry{2013--2014}{Ayudante de Ingenieria de Software}{Espol}{Guayaquil}{}
{Modelamiento de Software en UML, Planificación de proyectos de Software}

\iffalse
\newline{}%
Detalle de logros:%
\begin{itemize}%
\item Logro 1;
\item Logro 2, con sub-logros:
  \begin{itemize}%
  \item Sub-logro (a);
  \item Sub-logro (b), con sub-sub-logros (¡evite hacer esto!);
    \begin{itemize}
    \item Sub-sub-logro i;
    \item Sub-sub-logro ii;
    \item Sub-sub-logro iii;
    \end{itemize}
  \item Sub-logro (c);
  \end{itemize}
\item Logro 3.
\end{itemize}}
\fi
\cventry{2014--2015}{Asistente de TICS 3}{UNA E.P. }{Guayaquil}{}
{Desarollo de una Aplicacion WEB con tecnología JAVA EE para administrar inventario, generar ordenes de compra y realizar transferencias entre bodegas de productos \newline
}
\cventry{2015}{Analista de Sistemas}{Siconel}{Guayaquil}{}
{Desarollo de software con Odoo V8 y V7, Implementación de un Punto de Venta para linea de tiendas de ropa (GetIn) este punto de venta funciona actualmente en varios locales comerciales de la ciudad e Guayaquil}
\cventry{2017 (Actual)}{Analista de Sistemas}{LogaSystems}{Guayaquil}{}
{Soy responsable de la migración de la aplicación principal del negocio de PHP(Laravel) a Java EE7 }












\section{Proyectos Realizados}
\cvitem{JSF-Jasper}{Proyecto personal en el cual se integro la librería Jasper Reports en una
aplicación web desarrollada en JSF con Primefaces, el proyecto muestra como
a partir de una tabla de datos generada con Primefaces, se pueden exportar los
mismos como un reporte en formatos PDF, WORD, EXCEL, y POWER POINT.\newline
\href{https://github.com/jumarome/jsfJasperReports}{https://github.com/jumarome/jsfJasperReports}
}

\cvitem{JSF-Shiro}{Proyecto personal en el cual se integro la librería Apache Shiro a una aplicación
web en JSF. el proyecto muestra como manejar el control de acceso y la
autenticación dividiendo el esquema de seguridad en usuarios, roles y permisos
\href{https://github.com/jumarome/shirofaces}{https://github.com/jumarome/shirofaces}
}

\clearpage
\section{Referencias}
\cvitem{\textbf{Nombre}}{MSc. Carlos Mera Gómez}
\cvitem{\textbf{Cargo}}{MSc Internet Software Systems Docente Espol}
\cvitem{\textbf{Empresa}}{ESPOL}
\cvitem{\textbf{Telefono}}{0993644395}
\cvitem{\textbf{Correo}}{cjmera@espol.edu.ec\newline}

\cvitem{\textbf{Nombre}}{Dr. Mario Polit}
\cvitem{\textbf{Cargo}}{Cirujano oftalmólogo}
\cvitem{\textbf{Empresa}}{Alta-Vision}
\cvitem{\textbf{Telefono}}{0994139568}
\cvitem{\textbf{Correo}}{mpolit@hotmail.com\newline}


\cvitem{\textbf{Nombre}}{Dr.Luis Sarrazin}
\cvitem{\textbf{Cargo}}{Oftalmologo}
\cvitem{\textbf{Empresa}}{OMNI HOSPITAL}
\cvitem{\textbf{Telefono}}{0991968052}
\cvitem{\textbf{Correo}}{lusamo@hotmail.com\newline}


\cvitem{\textbf{Nombre}}{Soc. Edmundo Aguilar}
\cvitem{\textbf{Cargo}}{Asistente Tecnico Administrativo}
\cvitem{\textbf{Empresa}}{ESPOL}
\cvitem{\textbf{Telefono}}{2383163}
\cvitem{\textbf{Correo}}{edmundoaguilarnavarro@gmail.com\newline}


\renewcommand{\listitemsymbol}{-~}            % para cambiar el simbolo para las listas


% Las publicaciones tomadas de un archivo de BibTeX sin usar multibib\renewcommand*{\bibliographyitemlabel}{\@biblabel{\arabic{enumiv}}}

\nocite{*}
\bibliographystyle{plain}
\bibliography{publications}                   % 'publications' es el nombre del archivo BibTeX

% Las publicaciones tomadas de un archivo BibTeX usando el paquete multibib
%\section{Publicaciones}
%\nocitebook{book1,book2}
%\bibliographystylebook{plain}
%\bibliographybook{publications}              % 'publications' es el nombre del archivo BibTeX
%\nocitemisc{misc1,misc2,misc3}
%\bibliographystylemisc{plain}
%\bibliographymisc{publications}              % 'publications' es el nombre del archivo BibTeX

%\clearpage\end{CJK*}                          % si esta redactando su CV en chino usando CJK, \clearpage es requerido por fancyhdr para que funcione correctamente con CJK, aunque esto eliminara la numeracion de pagina al dejar \lastpage como no definido
\end{document}


%% fin del archivo `template-es.tex'.
